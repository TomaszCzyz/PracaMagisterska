%%% & --translate-file=cp1250pl
%% ************ AKADEMIA GÓRNICZO-HUTNICZA W KRAKOWIE **************
%% ***************** Wydział Matematyki Stosowanej ***************** 
%% ****************** PRACA MAGISTERSKA w LaTeX-u ******************
%%    autor: Tomasz Czyż
%%    Copyright (C) 2003 by ------
%% ************************* Plik główny *************************

%%
%% ======== PREAMBUŁA ======== 
%%
\documentclass[oik, pdftex, robocza, man]{mgrwms}

\usepackage[utf8]{inputenc}  % opcja latin2 dla Linux
\usepackage{amsmath}           % łatwiejszy skład matematyki
\newcommand\numberthis{\addtocounter{equation}{1}\tag{\theequation}}
\usepackage{amssymb} 
\usepackage{latexsym}
\usepackage{amsthm}
\usepackage{enumerate}

\usepackage{color}

\usepackage[polish]{babel}
\usepackage[OT4]{fontenc}
\usepackage{polski}
\allowdisplaybreaks
%% <<<< BiBTeX >>>>
% \bibliographystyle{ddabbrv}
% \nocite{*}

\begin{document}
%%
%% ======== METRYCZKA PRACY ========
%%
\title{ \LARGE Aproksymacja funkcji kawałkami regularnych przy użyciu informacji dokładnej i niedokładnej}
\author{Tomasz Czyż}
\promotor{dr Maciej Goćwin}
\nralbumu{290565}
\maketitle

\slowakluczowe{słowa kluczowe}
\keywords{keywords}
%%
%% ======== MAKRA ========
%%
%-> Miejsce na nasze makra (jedno z wielu ;). Lepszym pomysłem będzie jednak 
%-> umieszczenie ich w osobnym pliku i wczytanie poleceniem \input
%%
\newtheorem{thm}{\indent Twierdzenie}[chapter]
\newtheorem{lemma}[thm]{\indent Lemat}
\newtheorem{cor}[thm]{\indent Wniosek}
\newtheorem{obs}[thm]{\indent Obserwacja}
\newtheorem{uw}[thm]{\indent Uwaga}
\newtheorem{df}[thm]{Definicja}
\newcommand{\E}{\mathbb{E}}
\newcommand{\R}{\mathbb{R}}
\newcommand{\Pra}{\mathbb{Pra}}

\makeatletter
\newcommand*{\defeq}{\mathrel{\rlap{%
                     \raisebox{0.3ex}{$\m@th\cdot$}}%
                     \raisebox{-0.3ex}{$\m@th\cdot$}}%
                     =}
\let\c@table\c@figure
\makeatother

%%
%% ======== SPIS TREŚCI ========
%%
\tableofcontents
%%
%% ======== STRESZCZENIE PRACY (POLSKIE) ========
%%

\begin{streszczenie}
    Streszczenie
\end{streszczenie}

%%
%% ======== STRESZCZENIE PRACY (ANGIELSKIE) ========
%%

\begin{abstract}
    Abstract
\end{abstract}

%%
%%
%% ======== GŁÓWNA CZĘŚĆ PRACY ========
%%
%%

%%
%% ==== WSTĘP ====
%%

\begin{wstep}    % ew. \begin{wstep}[Wprowadzenie]
    Celem niniejszej pracy jest analiza zachowania różnych algorytmów aproksymujących funkcje kawałkami gładkie przy użyciu informacji dokładniej i niedokładniej.
    Pierwszy z omawianych algorytmów został przedstawiony w pracy \cite{PoA}, gdzie rozważane są funkcje klasy $F^{\infty}_r$. Zakładamy, że zarówno sama funkcja $f: [0,T] \longrightarrow \R$ może być nieciągła, jak i jej pochodna, począwszy od rzędu możliwe większego od pierwszego. Czyli, dla przykładu, $f$ może być dwa razy różniczkowalna na $[0, T]$ i $f^{(3)}(s)$ może nie istnieć w jakimś punkcie $s$. Ponadto, $f$ może mieć skończenie wiele punktów osobliwych; ich ilość i położenie jest nieznane. Dodatkowo, algorytm używa $n$ wartości funkcji w punktach $x_1, \ldots x_n$ jako jedyne dostępne informacje o funkcji $f$, a w przypadku algorytmu adaptacyjnego dopuszczamy, że wybór $x_j$ zależy od $f(x_1), \ldots, f(x_{j-1})$.
    W wymienionej pracy do znalezienia optymalnego algorytmu nieadaptacyjnego i adaptacyjnego w najgorszym przypadku oraz w przypadku asymptotycznym do mierzenia błędu stosowana jest m.in. norma $L^p (1 \leq  p < \infty)$
    \end{wstep}

%%
%% ==== ROZDZIAŁ 1 ====
%%

\chapter{Rozdział 1}

Jakaś zmiana

\mgrclosechapter

%%
%% ==== ROZDZIAŁ 2 ====
%%

%%
%% ======== DODATKI ========
%%
% \appendix
% \chapter{----}
%%
%-> Treść dodatku A
%%
% \mgrclosechapter

%%
%% ======== BIBLIOGRAFIA ========
%%

%% <<<< BiBTeX >>>>
% \bibliography{<pliki bib>} 
%%
\begin{thebibliography}{88}

    \bibitem{PoA}
    L. Plaskota, G. W. Wasilkowski, Y. Zhao, 
    \emph{The power of adaption for approximating functions with singularities}, Mathematics Of Computation 77
    2008, p. 2309–2338

    \bibitem{UA}
    L. Plaskota, G. W. Wasilkowski, 
    \emph{Uniform approximation of piecewise r-smooth and globally continuous functions}, SIAM Journal on Numerical
    Analysis, Vol. 47, No. 1 (2008/2009)

    \bibitem{CoDF}
    B. Kacewicz, P. Przybyłowicz, 
    \emph{Complexity of the derivative-free solution of
    systems of IVPs with unknown singularity hypersurface}, Journal of Complexity
    
    \bibitem{AoP}
    P. M. Morkisz, L. Plaskota, 
    \emph{Approximation of piecewise Hölder functions from inexact information}, Journal of Complexity

\end{thebibliography}

\end{document}
