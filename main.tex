%%% & --translate-file=cp1250pl
%% ************ AKADEMIA GÓRNICZO-HUTNICZA W KRAKOWIE **************
%% ***************** Wydział Matematyki Stosowanej ***************** 
%% ****************** PRACA MAGISTERSKA w LaTeX-u ******************
%%    autor: Tomasz Czyż
%%    Copyright (C) 2003 by ------
%% ************************* Plik główny *************************
%%
%-> Opcje klasy: 
%->  - man -> Wersja "meska" pracy :) Kwestia jednej litery w oświadczeniu. Domyślnie jest
%->           w wersji "damskiej".
%->  - robocza -> Produkuje wersję roboczą (nagłówek wzbogacony o datę kompilacji oraz
%->               nazwisko autora w stopce) 
%->  - pdftex -> Obowiązkowa przy kompilacji pdfLaTeX-em.
%->  - mfu, oik, pit, opt -> Specjalności, patrz niżej.
%->  - twoside -> standardowa w klasie mwbk. Posłuży nam do przygotowania wersji dwustronnej pracy
%%
%% ======== PREAMBUŁA ======== 
%%
\documentclass[oik, pdftex, robocza, man]{mgrwms}

\usepackage[utf8]{inputenc}  % opcja latin2 dla Linuxa
\usepackage{amsmath}           % łatwiejszy skład matematyki
\newcommand\numberthis{\addtocounter{equation}{1}\tag{\theequation}}
\usepackage{amssymb} 
\usepackage{latexsym}
\usepackage{amsthm}
\usepackage{enumerate}

\usepackage{color}

\usepackage[polish]{babel}     % pakiet 'babel' koniecznie z pakietem 'fontenc'!
\usepackage[OT4]{fontenc}
\usepackage{polski}
\allowdisplaybreaks
%% <<<< BiBTeX >>>>
% \bibliographystyle{ddabbrv}
% \nocite{*}

\begin{document}
%%
%% ======== METRYCZKA PRACY ========
%%
\title{ \LARGE Tytuł}
\author{Autor}
\promotor{dr xxx}
\nralbumu{000000}
\maketitle

\slowakluczowe{słowa kluczowe}
\keywords{keywords}
%%
%% ======== MAKRA ========
%%
%-> Miejsce na nasze makra (jedno z wielu ;). Lepszym pomysłem będzie jednak 
%-> umieszczenie ich w osobnym pliku i wczytanie poleceniem \input
%%
\newtheorem{thm}{\indent Twierdzenie}[chapter]
\newtheorem{lemma}[thm]{\indent Lemat}
\newtheorem{cor}[thm]{\indent Wniosek}
\newtheorem{obs}[thm]{\indent Obserwacja}
\newtheorem{uw}[thm]{\indent Uwaga}
\newtheorem{df}[thm]{Definicja}
\newcommand{\E}{\mathbb{E}}
\newcommand{\R}{\mathbb{R}}
\newcommand{\Pra}{\mathbb{Pra}}

\makeatletter
\newcommand*{\defeq}{\mathrel{\rlap{%
                     \raisebox{0.3ex}{$\m@th\cdot$}}%
                     \raisebox{-0.3ex}{$\m@th\cdot$}}%
                     =}
\let\c@table\c@figure
\makeatother

% \input{----}

%%
%% ======== SPIS TREŚCI ========
%%
\tableofcontents
%%
%% ======== STRESZCZENIE PRACY (POLSKIE) ========
%%

\begin{streszczenie}
    Streszczenie
\end{streszczenie}

%%
%% ======== STRESZCZENIE PRACY (ANGIELSKIE) ========
%%

\begin{abstract}
    Abstract
\end{abstract}

%%
%% ======== GŁÓWNA CZĘŚĆ PRACY ========
%%

%%
%% ==== WSTĘP ====
%%

\begin{wstep}    % ew. \begin{wstep}[Wprowadzenie]
    Wstep
\end{wstep}

%%
%% ==== ROZDZIAŁ 1 ====
%%

\chapter{Rozdział 1}

Lotem ipsum

\mgrclosechapter

%%
%% ==== ROZDZIAŁ 2 ====
%%

%%
%% ======== DODATKI ========
%%
% \appendix
% \chapter{----}
%%
%-> Treść dodatku A
%%
% \mgrclosechapter

%%
%% ======== BIBLIOGRAFIA ========
%%

%% <<<< BiBTeX >>>>
% \bibliography{<pliki bib>} 
%%
\begin{thebibliography}{88}
    %-> Treść bibliografii. 
    % \bibitem{<Klucz>}
\end{thebibliography}

\end{document}
